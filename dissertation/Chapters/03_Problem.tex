\chapter{The problem and its challenges}

"The problem with bad security is that it looks just like good security. You can’t tell the difference by looking at the finished product. Both make the same security claims; both have the same functionality. (...) Both might use the same protocols, implement the same standards, and have been endorsed by the same industry groups. Yet one is secure and the other is insecure." Schneier


%Pedaços de texto que posso usar mais tarde
"A resilient consensus protocol is only useful when it continues to deliver the intended service even under adversarial influence on the nodes and the network. Detailed analysis and formal argumentation are necessary to gain confidence that a protocol achieves its goal. In that sense, distributed consensus protocols resemble cryptosystems and other security mechanisms: they require broad agreement on the underlying assumptions, detailed security models, formal reasoning, expert judgement, open validation and wide spread public discussion. 

Verifying and establishing trust in the security, consistency and liveness semantics of a blockchain protocol is challenging.
A security solution should come with a clearly stated security model and trust assumption, under which the solution should satisfy its goal. This is widely accepted today; it prompts the question of how to validate that the solution satisfies its goal.

The experimental validation of a security solution in the information technology space fails very often because no experiment can exhaustively test the solution in all scenarios permitted by the model. In a way, experimentation can only demonstrate the failure of a security mechanism.

Therefore one needs to apply mathematical reasoning and formal tools to reason why the solution would remain secure under any scenario permitted by the stated trust assumption. Without such reasoning, security claims remain vague.

In a “provably secure” solution, an attack on the stated goal of the solution can be turned efficiently into a violation of some underlying assumption."
\citep{wild}







\iffalse

	         The problem and its challenges.

	\section{Proposed Approach - solution}
	In this section, it is presented various ways to display an image.
     \subsection{System Architecture}
     A block diagram of the planned system / approach

	Here we have an example of inserting an image between the text paragraphs.
	\begin{center}
		\includegraphics[width=0.2\textwidth]{img/mei-logo-03.jpg}
	\end{center}

	\begin{wrapfigure}{r}{0.25\textwidth}	
		\includegraphics[width=0.2\textwidth]{img/mei-logo-03.jpg}
	\end{wrapfigure}
	Here we have how an image can be wrapped into the text without having surronding space, and takin advantage of the space to be disposed on the side, without breaking the text readability.

	This approach also benefits from the fact that the text will be related implicitly to the image on its side, although the it should be referenced on the text anyway, otherwise, it should be consulting to perceive to which paragraph the image is related to.

	Here is how we place an image as floating body.
	Take in attention that the image is displayed on the next page, because there's no more room in this page.
	\begin{figure}
	\begin{center}
		\includegraphics[width=0.5\textwidth]{img/mei-logo-03.jpg}
	\end{center}
	\caption{caption}
	\end{figure}



	You can also use an image as an icon, eg.~\href{http://mei.di.uminho.pt}{\includegraphics[width=0.05\textwidth]{img/mei-logo-03.jpg}}, in the main tex.
	Click on it to visit the website. It is also listed in the list of terms.
	Another example of an item to appear in the term index: %\gls{um} (needs \Makeindex)
	
	\fi